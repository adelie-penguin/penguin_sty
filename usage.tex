% vim: set foldmethod=marker :
%
% vimで折畳機能を利用しない場合は,
% 本ファイル1行目のコメント(モードライン)を削除してください
%

%==============================================
\section{スタイルファイルの利用方法}
\label{sec:usage}
%==============================================

本文章は,usage.texファイルに記載されています.
main.texからは$\backslash$inputを利用して挿入されていますので,
本文章を削除する際には,main.texの該当行をコメントアウトしてください.

本スタイルファイルでは,Web検索などで見受けられる一般的なlatex構文を簡易化したコマンドを
いくつか用意してあります.
必要に応じて第\ref{sec:master}章のsubsectionを確認するようにお願いします.
とはいえ,TeXによる文書作成に慣れている方は,本文を読み飛ばして
角丸の囲み線の中に書かれたTeXコードのみ見てもらえれば十分かと思います.

%----------------------------------------------{{{
\subsection{コンパイルするために必要な準備}
%----------------------------------------------
本スタイルファイルをそのままコンパイルするためには,
styディレクトリ内にpenguin.bstとjlisting.styの2ファイルを用意する必要があります.
2つのstyファイルは,元ファイルが再配布可能かどうか判断できなかったため,
本スタイルファイルで非公開としています.
(研究室メンバーにはすでに配布してありますので,それらのファイルをコピーしてください)

jlisting.styはWeb公開されているスタイルファイルですので,
ダウンロードサイトから入手することができます.
一方,penguin.bstは情報処理学会のipsjunsrt.bstを改造したファイルを想定しています.
penguin.bstがない場合は,別のbstファイルを利用してください.

参考文献ファイルのbibfile.bibはpenguin.bstを想定した内容なので,
penguin.bst以外のbstファイルを利用するとpbibtexコマンドでコンパイルエラーが出るかもしれません.
その場合は,bibfile.bibの内のテキストをすべて削除すると
(参考文献の引用は印字されませんが)コンパイルできるようになるはずです.
%}}}
%----------------------------------------------{{{
\subsection{コンパイル}
%----------------------------------------------
makefileを用意してあるので,makeコマンドを実行すればPDFファイルを自動的に作成してくれるはずです.
ただし,makeコマンド内では,参考文献の有無に関係なくpbibtexコマンドが実行されます.
このため,TeXファイルが正しく記述されていても
参考文献の引用が無い場合などにpbibtexコマンドでエラーが生じることがあります.
基本的には,makeコマンドの実行結果としてmain.pdfという名前のPDFファイルが生成されているならば,
途中で生じたエラーを無視しても問題ないはずです.

makeコマンドでは,コンパイルのコメント表示を抑制してあります.
もしコンパイルによってPDFファイルの作成に失敗した場合は,
「platex main」コマンドを別途実行してエラー箇所を特定してください.
%}}}
%=============================================={{{
\section{スタイルファイルの固有設定}
\label{sec:master}
%==============================================
本スタイルファイルには簡易的なコマンドをいくつか用意してありますので,それらについて紹介します.

%----------------------------------------------{{{
\subsection{数式}
%----------------------------------------------
数式の印字には様々な表記方法がありますが,
なるべく$\backslash$begin$\{$align$\}$コマンドを利用したalign環境を利用しましょう.
align環境は,equatoinなどの多くの数式環境に対して上位互換となっています.
\eqref{eq:example}に数式の例を示します.

数式の参照には$\backslash$eqrefコマンドを利用してください.
本スタイルファイルでは,
$\backslash$eqrefコマンドを使ってlabelを参照しないと式番号が表示されませんので,
「式番号が表示されない!」と焦らないでください.
%
\begin{align}
	5 = 2 + 3
	\label{eq:example}
\end{align}
%
\begin{lstbox}{\eqref{eq:example}の数式を表示するTeX記法}
\begin{minilst}
\begin{align}
	5 = 2 + 3
	\label{eq:1}
\end{align}
\end{minilst}
\end{lstbox}
%}}}
%----------------------------------------------{{{
\subsection{図}
%----------------------------------------------
$\backslash$fig, $\backslash$doublefig, $\backslash$subfigは,
figureディレクトリに保存された画像ファイルを文書内に表示するためのコマンドです.
yukkuri.jpg画像を横幅4.5cmで貼り付ける例を\figref{fig:yukkuri}に示します.
eps以外の画像を貼り付けたい場合は,
オプション引数($[\ ]$括弧の部分)で画像ファイルの拡張子を設定してください.
epsファイルの場合はオプションを省略して記述できます.
もちろん,オプション引き数を「eps」と明示的に記述することも可能です.

下記のコマンドを利用する場合は,
図番号の参照に$\backslash$figrefコマンドを利用してください.
参照する際のラベルは,図のファイル名に基づいて設定されますので,
$\backslash$figref$\{$fig:図のファイル名$\}$と記述してください.

下記のコマンドでは,ファイル名をラベル名として利用しています.
この都合上,同じ名前の画像ファイルを2回以上貼り付けることはできませんので
注意してください.
(一応,複製ファイルを別名で作成すれば解決できます.)
%
\begin{lstbox}{図を貼るコマンドの引数}
\begin{minilst}
\fig{図のタイトル}{図のファイル名}{図の横幅}
\doublefig{左図タイトル}{左図ファイル名}{左図横幅}{図間距離}{右図タイトル}{右図ファイル名}{右図横幅}
\subfig{全体タイトル}{左図タイトル}{左図ファイル名}{左図横幅}{図間距離}{右図タイトル}{右図ファイル名}{右図横幅}
\end{minilst}
\end{lstbox}
%
\fig[jpg]{ゆっくりしていってね}{yukkuri}{4.5} %画像の貼付
%
\begin{lstbox}{\figref{fig:yukkuri}を貼るためのTeX記法}
\begin{minilst}
\fig[jpg]{ゆっくりしていってね}{yukkuri}{4.5}
\end{minilst}
\end{lstbox}

%}}}
%----------------------------------------------{{{
\subsection{表}
%----------------------------------------------
表を挿入する場合は,特にこだわりがない限りは
独自定義したttable環境を利用してください.
下記に\tabref{tb:example}を表示するためのTeX記法を2種類例示しますので,
ttable環境の有無による記法の変化を確認してみてください.

表の参照には$\backslash$tabrefコマンドを利用してください.
参照する際のラベルは,引数のラベル名に基づいて設定されますので,
$\backslash$tabref$\{$tb:ラベル名$\}$と記述してください.
%{{{
\begin{lstbox}{\tabref{tb:example}のテーブルを表示するTeX記法(ttable環境利用)}
\begin{minilst}
\begin{ttable}[t]{静的配列の実行時間[ms]}{example}{l|r|r|r}{0.3}
            & array & pointer  & scalar \\\hline
with avx    & 4.0   & 3.2      & 2.5    \\
without avx & 4.8   & 4.1      & 2.5    \\\hline
\end{ttable}
\end{minilst}
\end{lstbox}
%}}}
%{{{
\begin{lstbox}{\tabref{tb:example}のテーブルを表示するTeX記法(一般的な記法)}
\begin{minilst}
\begin{table}[t]
\caption{静的配列の実行時間[ms]}
\label{tb:example}
\tbspace{0.3}	%セルの左右余白サイズ[cm]
\begin{tabular}{l|r|r|r}
\hline\hline
            & array & pointer  & scalar \\\hline
with avx    & 4.0   & 3.2      & 2.5    \\
without avx & 4.8   & 4.1      & 2.5    \\\hline
\end{tabular}
\end{table}
\end{minilst}
\end{lstbox}
%}}}
%{{{
\begin{ttable}[t]{静的配列の実行時間[ms]}{example}{l|r|r|r}{0.3}
            & array & pointer  & scalar \\\hline
with avx    & 4.0   & 3.2      & 2.5    \\
without avx & 4.8   & 4.1      & 2.5    \\\hline
\end{ttable}
%}}}

%}}}
%----------------------------------------------{{{
\subsection{ソースコード}
%----------------------------------------------
ソースコードの表示にはlstlisting環境を用いることができます.
\figref{fig:hello}を表示するためのTeXコードの例を示します.

lstlisting環境内であれば,
コメントアウトされた部分もそのまま反映できますので,
C言語コードをコピペするだけで問題なく表示できるはずです.
ただし,スタイルファイル内で「折り返し」を設定してあるので,
「//」を利用した長文コメントには多少の編集が必要となります.

実行コマンドのように,図番号を付けるほどでもない内容には,
lstbox環境とminlist環境を利用した記述を使ってください..
%{{{
\begin{lstbox}{\figref{fig:hello}のソースコードを表示するためのTeX記法}
\begin{minilst}
\begin{figure}[ht] %横にぶち抜きの図を作るときは{figure*}にする
\begin{lstlisting}
__global__ void function(double *A, int size)
{
	//引数を使わない不届きなプログラム
	printf("Hello World!\n");	/* コメント */
}	
\end{lstlisting}
\caption{プログラム例}
\label{fig:hello}
\end{figure} %横にぶち抜きの図を作るときは{figure*}にする
\end{minilst}
\end{lstbox}
%}}}
%{{{
\begin{figure}[ht] 
\begin{lstlisting}
__global__ void function(double *A, int size)
{
	//引数を使わない不届きなプログラム 
	printf("Hello World!\n");	/* コメント */
}
\end{lstlisting}
\caption{プログラム例}
\label{fig:hello}
\end{figure} 
%}}}
%{{{
\begin{lstbox}{図番号のない簡易プログラム}
\begin{minilst}
\begin{lstbox}{コードのタイトルを書く}
\begin{minilst}
	<ここにソースコード>
\end{minilst}
\end{lstbox}
\end{minilst}
\end{lstbox}
%}}}
%}}}
%----------------------------------------------{{{
\subsection{引用}
\label{sec:ref}
%----------------------------------------------
表の参照には$\backslash$tabref,
図の参照には$\backslash$figref,式の参照には$\backslash$eqrefを用いて記述します.
参考文献は,「文献\Cite{test2}では~」というように名詞として引用する場合は$\backslash$Cite,
文節や文全体の内容が引用である場合は$\backslash$citeを利用してください
\cite{test1}\cite{test3}.

\begin{lstbox}{参照のTeX記法}
\begin{minilst}
\figref{fig:penguin}は図の参照である.
\tabref{tb:penguin}は表の参照である.
\eqref{eq:penguin}は式の参照である.
文献\Cite{penguin}は参考文献を参照する\cite{penguin}.
\end{minilst}
\end{lstbox}

本フォーマットでは,図表の参照箇所を強調するために,
初めて参照される図表番号を太字で表示します.
強調するのは初回のみなので,2回目以降の参照箇所は太字になりません.
また,引用のない数式には式番号が表示されないように設定してあります.

%}}}
%----------------------------------------------{{{
\subsection{その他}
%----------------------------------------------

%----------------------------------------------
\subsubsection{usage.texの文章について}
%----------------------------------------------
本稿の第\ref{sec:usage}章以降の文章は,usage.texに記述してあります.
慣例的に,TeXフォーマットは,main.tex自身をREADME文章とすることで
フォーマットを周知する文化がありますが,
ちょっと長すぎるなぁと思ったのでファイル分割してinputする形を採用しています.

%----------------------------------------------
\subsubsection{謎のコメント「$\{\{\{$」}
%----------------------------------------------
usage.texファイル中にある「$\{\{\{$」や「$\%\}\}\}$」は,
vimで折畳機能を利用するためのマーカーです.
latexでは$\%$以降の記述がコメントとなりますので,
マーカー部分の有無がコンパイル後のPDFファイルに影響を与えることはありません.
とはいえ,vim以外のエディタで閲覧すると
無駄なコメント行が多くあるように見えてしまうかもしれません.
配布時のタイミングでは「$\{\{\{$」や「$\}\}\}$」の文字列を
vim用のマーカーでしか利用していませんので,
気になる場合は文字列変換で消去してください.

%----------------------------------------------
\subsubsection{文字の彩色}
%----------------------------------------------
\red{赤字}や\blue{青字}での印刷コマンドを用意しました.
白黒印刷では灰色になってしまいますが,添削などには便利かと思われます.
これらのコマンドは,color.styの機能を利用した自作コマンドで定義されています.
赤や青以外の色を利用したい場合は各自で調査をお願いします.
\begin{lstbox}{文字に色を付けるTeX記法}
\begin{minilst}
 \red{赤字}
 \blue{青字}
\end{minilst}
\end{lstbox}

%}}}

%}}}
